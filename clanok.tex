% Metódy inžinierskej práce

\documentclass[10pt,oneside,slovak,a4paper]{article}

\usepackage[slovak]{babel}
%\usepackage[T1]{fontenc}
\usepackage[IL2]{fontenc} % lepšia sadzba písmena Ľ než v T1
\usepackage[utf8]{inputenc}
\usepackage{graphicx}
\usepackage{url} % príkaz \url na formátovanie URL
\usepackage{hyperref} % odkazy v texte budú aktívne (pri niektorých triedach dokumentov spôsobuje posun textu)

\usepackage{cite}
%\usepackage{times}

\pagestyle{headings}

\title{Vplyv spoločenských hier na mentálne zdravie\thanks{Semestrálny projekt v predmete Metódy inžinierskej práce, ak. rok 2022/23, vedenie: Ing. Vladimír Mlynarovič, PhD.}} % meno a priezvisko vyučujúceho na cvičeniach

\author{René Klačan\\[2pt]
	{\small Slovenská technická univerzita v Bratislave}\\
	{\small Fakulta informatiky a informačných technológií}\\
	{\small \texttt{xklacan@stuba.sk}}
	}

\date{\small 18. október 2022}



\begin{document}

\maketitle
\begin{abstract}
\ldots
\end{abstract}



\section{Úvod}

V článku sa budeme zaoberať vplyvom stolných hier na ľudskú mentalitu~\ref{vplyv_hier_na_mentalitu}. Ďalej sa dozvieme o diferencii týchto vplyvom na odborníkov ohľadom neurovedy, adulescentoch a detí vrátane trpiace mentálnym postihnutím. Budeme môcť postrehnúť zmeny v kognitívnych funkcii človeka zapríčinené hraním stolných hier, ktoré budú doložené niekoľkými výskumami.


\section{Aplikácia stolných hier} \label{aplikacie_hier}
Ná základe výsledkov štúdii \cite{Nakao:BG}, ktoré skúmali tradičné a netradičné stolné hry, bolo zistené, že majú pozitívny vplyv na prevenciu voči kognitívnym poruchám ako u starších jedincov, tak aj u detí.
Predpokladá sa, že hranie stolných hier môže byť efektívnym a zábavným prostriedkom pre prevenciu vyššie spomínaným nepriaznivých účinkov \cite{Edu:GFH}
Je taktiež dôležité poznamenať, že stolné hry môže byť hrané bez použitia jazyka. V porovnaní s terapeutickými sedeniami, ktoré pozostávajú hlavne z používania jazyka ako hlavného dorozumievacieho média. Pri ľuďoch, ktorí nemajú dostatočne rozvinuté jazykové funkcie toto môže byť problém, avšak vďaka stolným hrám je možné tento problém predísť \cite{Charlier:2013}.

\section{Vplyv hier na rozvoj kognitívnych funkcií} \label{hry_kognitivne_funkcie}
Štúdie \cite{} použili stolné hry go, šach a ska, čo nie sú vzdelávacie hry, ale naopak abstraktné strategické hry.

\begin{itemize}
    \item Go
    	\begin{itemize}
        	\item {Pri tejto hre, štúdia \cite{Aydin:2015} zistili, že pri senioroch, ktorí trpia kognitívnym poklesom, bolo preukázané zlepšenie pozornosti a práci pamäte pri pravidelnom hraní tejto hry.}
    	\end{itemize}
    \item Ska
        \begin{itemize}
        	\item{Pri hraní tejto hry bolo preukázané \cite{Panphunpho:2013} pri senioroch rapídne zlepšenie kognitívnych funkcii. Zahŕňajúc zlepšenie pamäti, pozornosti a schopnosť strategizácie.}
    	\end{itemize}
    \item Šach
        \begin{itemize}
        	\item{Štúdie hodnotiace šach ukázali \cite{Sala:2015, Demily:2009}, že pravidelné hranie ľudí trpiacich schizofréniou vyústilo k zlepšeniu ich plánovacích a matematických schopností.}
    	\end{itemize}
\end{itemize}

\section{Iná časť} \label{ina}

Základným problémom je teda\ldots{} Najprv sa pozrieme na nejaké vysvetlenie (časť~\ref{ina:nejake}), a potom na ešte nejaké (časť~\ref{ina:nejake}).\footnote{Niekedy môžete potrebovať aj poznámku pod čiarou.}

Môže sa zdať, že problém vlastne nejestvuje\cite{Coplien:MPD}, ale bolo dokázané, že to tak nie je~\cite{Czarnecki:Staged, Czarnecki:Progress}. Napriek tomu, aj dnes na webe narazíme na všelijaké pochybné názory\cite{PLP-Framework}. Dôležité veci možno \emph{zdôrazniť kurzívou}.


\subsection{Nejaké vysvetlenie} \label{ina:nejake}

Niekedy treba uviesť zoznam:

\begin{itemize}
\item jedna vec
\item druhá vec
	\begin{itemize}
	\item x
	\item y
	\end{itemize}
\end{itemize}

Ten istý zoznam, len číslovaný:

\begin{enumerate}
\item jedna vec
\item druhá vec
	\begin{enumerate}
	\item x
	\item y
	\end{enumerate}
\end{enumerate}


\subsection{Ešte nejaké vysvetlenie} \label{ina:este}

\paragraph{Veľmi dôležitá poznámka.}
Niekedy je potrebné nadpisom označiť odsek. Text pokračuje hneď za nadpisom.



\section{Dôležitá časť} \label{dolezita}




\section{Ešte dôležitejšia časť} \label{dolezitejsia}




\section{Záver} \label{zaver} % prípadne iný variant názvu



%\acknowledgement{Ak niekomu chcete poďakovať\ldots}


% týmto sa generuje zoznam literatúry z obsahu súboru literatura.bib podľa toho, na čo sa v článku odkazujete
\bibliography{literatura}
\bibliographystyle{plain} % prípadne alpha, abbrv alebo hociktorý iný
\end{document}

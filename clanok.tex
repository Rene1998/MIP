% Metódy inžinierskej práce

\documentclass[10pt,oneside,slovak,a4paper]{article}

\usepackage[slovak]{babel}
%\usepackage[T1]{fontenc}
\usepackage[IL2]{fontenc} % lepšia sadzba písmena Ľ než v T1
\usepackage[utf8]{inputenc}
\usepackage{graphicx}
\usepackage{url} % príkaz \url na formátovanie URL
\usepackage{hyperref} % odkazy v texte budú aktívne (pri niektorých triedach dokumentov spôsobuje posun textu)

\usepackage{cite}
%\usepackage{times}

\pagestyle{headings}

\title{Vplyv spoločenských hier na mentálne zdravie\thanks{Semestrálny projekt v predmete Metódy inžinierskej práce, ak. rok 2022/23, vedenie: Ing. Vladimír Mlynarovič, PhD.}} % meno a priezvisko vyučujúceho na cvičeniach

\author{René Klačan\\[2pt]
	{\small Slovenská technická univerzita v Bratislave}\\
	{\small Fakulta informatiky a informačných technológií}\\
	{\small \texttt{xklacan@stuba.sk}}
	}

\date{\small 18. október 2022}

\begin{document}

\maketitle
\begin{abstract}
Tento článok má za cieľ priblížiť čitateľovi problematiku zaoberajúcu sa vplyvom spoločenských (stolných) hier na mentalitu človeka. Poukazuje na prevenciu voči mentálnym chorobám / disfunkciám ako u mladistvých, tak aj u detí. Celý článok je podložený niekoľkými štúdiami zaoberajúcich sa touto problematikou. Taktiež v práci poukazujeme na hry, ktoré majú preukázaný pozitívny vplyv na kognitívne funkcie človeka.
\end{abstract}

\section{Úvod}
V článku sa budeme zaoberať vplyvom stolných hier na ľudskú mentalitu~\ref{hry_kognitivne_funkcie}. Na prevenciu voči rôznym negatívnym vplyvom na kognitívne funkcie človeka a ich prevencii. Poukážeme na niekoľko preukázaných možností prevencie. A spomenieme niekoľko štúdii.

\section{Aplikácia stolných hier} \label{aplikacie_hier}
Ná základe výsledkov štúdii \cite{Nakao:BG}, ktoré skúmali tradičné a netradičné stolné hry, bolo zistené, že majú pozitívny vplyv na prevenciu voči kognitívnym poruchám ako u starších jedincov, tak aj u detí.
Predpokladá sa, že hranie stolných hier môže byť efektívnym a zábavným prostriedkom pre prevenciu vyššie spomínaným nepriaznivých účinkov \cite{Edu:GFH}
Je taktiež dôležité poznamenať, že stolné hry môže byť hrané bez použitia jazyka. V porovnaní s terapeutickými sedeniami, ktoré pozostávajú hlavne z používania jazyka ako hlavného dorozumievacieho média. Pri ľuďoch, ktorí nemajú dostatočne rozvinuté jazykové funkcie toto môže byť problém, avšak vďaka stolným hrám je možné tento problém predísť \cite{Charlier:2013}.

\section{Vplyv hier na rozvoj kognitívnych funkcií} \label{hry_kognitivne_funkcie}
Štúdie \cite{Panphunpho:2013} použili stolné hry go, šach a ska, čo nie sú vzdelávacie hry, ale naopak abstraktné strategické hry.

\begin{itemize}
    \item Go
    	\begin{itemize}
        	\item {Pri tejto hre, štúdia \cite{Aydin:2015} zistili, že pri senioroch, ktorí trpia kognitívnym poklesom, bolo preukázané zlepšenie pozornosti a práci pamäte pri pravidelnom hraní tejto hry.}
    	\end{itemize}
    \item Ska
        \begin{itemize}
        	\item{Pri hraní tejto hry bolo preukázané \cite{Panphunpho:2013} pri senioroch rapídne zlepšenie kognitívnych funkcii. Zahŕňajúc zlepšenie pamäti, pozornosti a schopnosť strategizácie.}
    	\end{itemize}
    \item Šach
        \begin{itemize}
        	\item{Štúdie hodnotiace šach ukázali \cite{Sala:2015, Demily:2009}, že pravidelné hranie ľudí trpiacich schizofréniou vyústilo k zlepšeniu ich plánovacích a matematických schopností.}
    	\end{itemize}
\end{itemize}

\subsection{Výskumy zameriavacie sa na zmenu kognitívnych funkcií človeka}
    V období od januára 2012 do augusta 2018 bolo publikovaných 83 štúdii, ktoré boli zamerané na problematiku využitia stolových hier \cite{Nakao:BG}.
    Z toho šesť bolo zameraných na vplyv stolných hier na základné mozgové činnosti.
    Ďalších päť bolo zameraných na na prevenciu voči demencii.
    Ďalej podľa článku, ktorý bol uverejnený v \emph{New England Journal of Medicine autorom Coyle})~\cite{doi:10.1056/NEJMp030051} bolo uvedené, že seniori starší než 75, ktorí sa zapájali vo voľnočasových aktivitách zahŕňajúc stolné hry resp. šach, sa preukázalo že symptómy demencie boli oneskorené v porovnaní so seniormi, ktorí sa týchto voľnočasových aktivít nezúčastňovali. Tohto výskumu za zúčastnilo skoro 500 dôchodcov po dobu piatich rokov. Seniori, ktorí sa zúčastňovali hrania stolných hier preukázali až o 35\% nižšiu šancu si vybudovať demenciu.

\section{Záver} \label{zaver} % prípadne iný variant názvu
Súčasný systematický prehľad ukázal, že hranie stolných hier majú pozitívny vplyv na rôzne kognitívne funkcie, rozvoj vzdelávacích funkcií a prevencií rôznych mentálnych disfunkciám alebo postihnutím. Tieto zistenia naznačujú, že stolné hry sa dajú využiť ako účinná doplnková terapia, môže prispieť k zlepšeniu mnohých aspektov človeka.



%\acknowledgement{Ak niekomu chcete poďakovať\ldots}


% týmto sa generuje zoznam literatúry z obsahu súboru literatura.bib podľa toho, na čo sa v článku odkazujete
\bibliography{literatura}
\bibliographystyle{plain} % prípadne alpha, abbrv alebo hociktorý iný
\end{document}
